\section{Išvados}

Šiame darbe buvo pristatytas vokselių vizualizavimo ir normavimo įrankis,
prieš tai pristatant pačius vokselius, jų vizualizavimo idėjas ir panaudojimą
įvairiose srityse.

Bekuriant įrankį buvo sukurtas realaus laiko transformavimo filtro modulis,
leidžiantis greitai ir paprastai pritaikyti vokselių permatomumo reikšmes
vizualizuojamam vaizdui. Taip pat buvo sugalvoti du nauji filtrai.

Pirmas, globalaus permatomumo filtras leidžia pritaikyti permatomumo slenksčio
modifikaciją, išvengiant grubaus peršokimo slenksčio prieigose, atsirasdavusio
naudojant tik slenksčio filtą. Re-zultatas -- subtilus vaizdo pagerėjimas,
geriausiai matomas, prieš tai pritaikius kontrastingą transformavimo filtrą.

Antras, apšvietimo sustiprinimo filtras leidžia išryškinti dėl permatomumo
susiliejančias tūrinio objekto detales. Pritaikius filtrą generuojamos
vizualizacijos pastebimai įgauna kokybiškesnį ir akiai malonesnį vaizdą.

Darbe taip pat pateikiama spalvotų kubų idėja, kuri nebuvo iki galo ištyrinėta
dėl tinkamos aparatūrinės įrangos stokos.

