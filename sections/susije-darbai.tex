\section{Susiję darbai}
\label{sec:susije_darbai}

Apie spindulių skleidimo idėjos pritaikymą tūriniams objektams vizualizuoti,
naudojant vokselius, kalbama jau 1988 m. Marc Levoy straipsnyje
„\textit{Display Of Surfaces from Volume Data}“ \cite{first}.

Jens Krüger ir Rüdiger Westermann 2003m. publikuotame „\textit{Acceleration
Techniques for GPU-based Volume Rendering}“ \cite{raycast} spindulių skleidimas
įgyvendinamas panaudojant spalvotų vienetinių kubų idėją, kuria aš daugiausiai
ir rėmiausi, kurdamas savo įrankį.

Dabartiniai darbai, naudojantys tūrinį spindulių skleidimą, daugiausia dėmesio
skiria tūrinių scenų su masiniu vokselių skaičiumi vizualizavimui. Tai Cyril
Crassin ir kiti autoriai su savo veikalais „\textit{Interactive GigaVoxels}“
\cite{gig2008} ir „\textit{GigaVoxels : Ray-Guided Streaming for Efficient and
Detailed Voxel Rendering}“ \cite{gig2009}. Juose pristatomas tūrinio skleidimo
algoritmas. Tačiau dėl to, kad jie bando dirbti su dideliu kiekiu duomenų --
jie privalo daryti tam tikras prielaidas (pvz.: kad dauguma objektų visiškai
nepermatomi).

Enrico Gobbetti ir kiti autoriai taip pat panaudojo spindulių skleidimą
masiniam vokselių vizualizavimui darbe „\textit{A single-pass GPU ray casting
framework for interactive out-of-core rendering of massive volumetric
datasets}“ \cite{other}. Kitaip nei prieš tai aptartų darbų autoriai, šie
visiškai atsisako dalinio permatomumo, tačiau dėl to leidžia sau naudotis
agresyvia vokselių blokelių atmetimo technika.

